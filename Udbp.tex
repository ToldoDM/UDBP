\documentclass[10pt]{article}

\usepackage[margin=1in]{geometry}
\setlength{\tabcolsep}{18pt}
\renewcommand{\arraystretch}{1.5}
\usepackage[table]{xcolor}
\usepackage{fontspec} 
\setmainfont{Arial}

\begin{document}
	\section{Bozza}
	si vuole realizzare una base di dati per l'organizzazione per una catena di cinema, chiamato per comodita' UIC Cinemas.
	
	La UCI Cinemas ha in possesso molteplici cinema situati in diverse citta'. Ogni cinema e' un multisala composto dalle 5 alle 10 sale ciascuna con posti che variano da 50 a 250 posti numerati. Ogni film proiettato varia  in lunghezza, genere, e cast (composto  da attori, sceneggiatori e produttori).
	La programmazione della proiezione dei film puo' variare di giorno in giorno (varia in base ai giorni della settimana, per esempio Martedi' vengono proiettati meno film rispetto al weekend) e comprende la proiezione un film in un cinema in una sala ad un ora specifica e nella durata totale della proiezone sono compresi dai 15 ai 30 min di pubblicita' (nelle programmazioni piu corpose vengono proiettati piu' film, di conseguenza viene ridotto la durata della pubblicita').
	Il consumatore per vedere un film deve acquistare un biglietto, tramite portale on-line o in biglietteria. Tale biglietto contiene all'interno informazioni inerenti a: cinema dove e la sala dove avviene la proiezione, il film di interesse, l'ora e il giorno di inizio film e il numero del posto [(il posto viene assegnato automaticamente al momento dell'acquisto)].
	
	
	\section{Abstract}
	
	La societa' UCI Cinemas e' una catena di cinema, nata in regno unito nel 1988 da una partnership tra Universal Studios e Paramount Pictures che nel corso degli anni si e' estesa sia nel campo nazionale, che nel campo internazionale. 
	continua...
	
	\section{Analisi dei requisiti}
	
	Il progetto vuole rappresentare una base di dati che permetta di gestire la programmazione dei film della catena UCI cinema.
			
	Essendo UCI cinemas una catena, e' necessario identificare la localita' di ciascun {\bf Cinema}:
	\begin{itemize}
		\item Nome cinema
		\item Numero sale
		\item Luogo, cosi' composto: via, citta', provincia, codice postale, regione, nazione
	\end{itemize}
	Per poter identificare una {\bf Sala} all'interno del cinema abbiamo bisogno delle sguenti informazioni:
	\begin{itemize}
		\item Numero sala (univoco all'interno del cinema)
		\item Numero posti
		\item Cinema di appartenenza
		\item Grandezza schermo (espresso in pollici)
	\end{itemize}
	All'interno di ogni sala viene proiettato un {\bf Film}:
	\begin{itemize}
		\item Titolo
		\item Durata
		\item Trama
		\item Genere
		\item Anno
	\end{itemize}
	Per conoscere il cast del film, c'e' bisogno di conoscere il nome e il ruolo delle persone che ne hanno preso parte, e si possono suddividere in:
	\begin{itemize}
		\item {\bf Attore}: persona che ha recitato nel film 
		\item {\bf Sceneggiatore}: persona che ha strutturato la sceneggiatura del film 
		\item {\bf Produttore}: persona che ha finanziato il film 
		\item {\bf Regista}: persona che ha diretto il film
	\end{itemize}		 
	Di conseguenza abbiamo bisogno di definire {\bf Lavoratore}:
	\begin{itemize}
		\item Nome
		\item Cognome
		\item Data di nascita
		\item Film (a cui ha partecipato)
	\end{itemize}
	Il cinema per emettere i biglietti non ha bisogno delle credenziali del cliente, di conseguenza il bilgietto non e' nomintivo. Il {\bf Biglietto} ha le seguenti caratteristiche:
	\begin{itemize}		
		\item Film (film di interesse)
		\item Orario inizio film
		\item Data proiezione
		\item Cinema (dove viene proiettato il film)
		\item Sala
		\item Posto (scelto dall'utente al momento dell'acquisto)
	\end{itemize}
 	Per massimizzare l'utilizzo delle sale viene creata una \textbf{Programmazione} in cui vengono specificati gli orari di inizio dei vari film per i vari giorni, quindi sono necessarie le seguenti informazioni:
 	\begin{itemize}
 		\item Cinema
 		\item Sala
 		\item Film
 		\item Ora inizio
 		\item Data film
 	\end{itemize} 
 	\subsection{Glossario dei termini}
 	\begin{tabular}{ |p{3cm}|p{4.5cm}|p{2.5cm}|p{3cm}|  }
 		%\hline
 		%\multicolumn{4}{|c|}{\textbf{Glossario dei termini}} \\
 		\hline
 		\rowcolor{lightgray}
 		\textbf{Termine} & \textbf{Descrizione} & \textbf{Sinonimi} & \textbf{Collegamenti} \\
 		\hline
 		Programmazione                         & Pianificazione orario inizio film all'interno del cinema                &                                                                                      & Biglietto, Film                                                             \\ \hline
 		Sala                                   & Stanza dove viene proiettato il Film                                    &                                                                                      & Biglietto, Cinema, Programmazione \\ \hline
 		Biglietto                              & Acquisto del cliente, che attesta la vendita del posto per un dato film &                                                                                      & Cliente, Film                                                               \\ \hline
 		Lavoratore                                & Persona singola che ha preso parte alla realizzazione del Film                  & Attore, Sceneggiatore, Produttore, Regista & Film                                                                        \\ \hline
 		Film                                   & Film presenti al cinema                                                 &                                                                                      & Programmazione, Biglietto        \\ \hline
 		Cinema                                   &    Impresa che offre il servizio cinematografico &                                                                                      & Programmazione, Biglietto, Sala        \\ \hline
 	\end{tabular}
 	\subsection{Strutturazione dei requisiti}
	\begin{tabular} { |p{16.8cm}| }
 		\hline
 		\rowcolor{lightgray}
 		\textbf{Frasi relative a "Cinema"} \\
 		\hline
 		I Cinema vengono identificati tramite: Nome cinema, Numero sale, Luogo (composto da via, citta', provincia, codice postale, regione, nazione) \\
 		\hline 		
 	\end{tabular}
 	\\\\\\
	\begin{tabular} { |p{16.8cm}| }
	 	\hline
	 	\rowcolor{lightgray}
	 	\textbf{Frasi relative a "Sala"} \\
	 	\hline
	 	Le Sale vengono identificate tramite: Numero sale (univoco all'interno del cinema), Numero posti, Cinema di appartenenza, Grandezza schermo (espresso in pollici) \\
	 	\hline 		
	\end{tabular} 
	\\\\\\
	\begin{tabular} { |p{16.8cm}| }
		\hline
		\rowcolor{lightgray}
		\textbf{Frasi relative a "Film"} \\
		\hline
		I Film vengono identificati tramite: Titolo, Durata, Trama, Genere, Anno \\
		\hline 		
	\end{tabular} 
 	\\\\\\
 	\begin{tabular} { |p{16.8cm}| }
 		\hline
 		\rowcolor{lightgray}
 		\textbf{Frasi relative a "Lavoratore"} \\
 		\hline
 		I Lavoratori vengono identificati tramite: Nome, Cognome, Data di nascita, Film (a cui ha lavorato) \\
 		\hline 		
 	\end{tabular} 
 	\\\\\\
 	\begin{tabular} { |p{16.8cm}| }
 		\hline
 		\rowcolor{lightgray}
 		\textbf{Frasi relative a "Attore"} \\
 		\hline
 		Persona che ha lavorato come attore nel film \\
 		\hline 		
 	\end{tabular} 
	\\\\\\
	\begin{tabular} { |p{16.8cm}| }
		\hline
		\rowcolor{lightgray}
		\textbf{Frasi relative a "Sceneggiatore"} \\
		\hline
		Persona che ha lavorato come sceneggiatore nel film \\
		\hline 		
	\end{tabular} 
	\\\\\\
	\begin{tabular} { |p{16.8cm}| }
		\hline
		\rowcolor{lightgray}
		\textbf{Frasi relative a "Produttore"} \\
		\hline
		Persona che ha lavorato come produttore nel film \\
		\hline 		
	\end{tabular} 
	\\\\\\
	\begin{tabular} { |p{16.8cm}| }
		\hline
		\rowcolor{lightgray}
		\textbf{Frasi relative a "Regista"} \\
		\hline
		Persona che ha lavorato come regista nel film \\
		\hline 		
	\end{tabular} 
	\\\\\\
	\begin{tabular} { |p{16.8cm}| }
		\hline
		\rowcolor{lightgray}
		\textbf{Frasi relative a "Biglietto"} \\
		\hline
		I Biglietti vengono idetificati tramite: Film, Orario, Data proiezione, Cinema, Sala, Posto \\
		\hline 		
	\end{tabular} 
	
	
\end{document}
